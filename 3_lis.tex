\section{Lenguaje Imperativo Simple}
  \subsubsection*{Gramática}
    $\begin{array}{llllll}
      \comm & ::= & \textbf{skip} \\
      && \fail \\
      && \var := \intexp \\
      && \comm\ ;\ \comm \\ 
      && \ife \boolexp\thene\comm\elsee\comm \\ 
      && \newvar{\var:=\intexp}{\comm} \\ 
      && \while{\boolexp}{\comm} \\
      && \catch{\comm}{\comm} \\
      && \mathbb{!} \intexp \\
      && \mathbb{?} \var \\
    \end{array}$

  \subsection*{\underline{Semántica Denotacional} (con fallas)}
    \begin{itemize}
      \item $(\ast)$: Se tranfiere el control a f SI NO HAY $\abort$
      \item $(+)$: Se tranfiere el control a f SOLO en caso de $\abort$
      \item $(\dag)$: Se tranfiere el control a f SIEMPRE
    \end{itemize}

    \pagebreak

    \begin{multicols}{2}
      \begin{eqnarray*}
        \semsAGU{\skipp} &=& \sigma \\
        \semsAGU{\fail} &=& \lang{\abort, \sigma} \\
        \semsAGU{v := e} &=& [\sigma | v : \semsAGU{e}] \\
        \semsAGU{\ifthenelse{b}{c_{0}}{c_{1}}} &=& \begin{cases}
                                                  \semsAGU{c_0} & \text{si } \semsAGU{b} \\
                                                  \semsAGU{c_1} & c.c \\
                                                \end{cases}\\
        \semsAGU{c_{0};c_{1}} &=& \semAGU{c_{1}}_{\ast} (\semsAGU{c_{0}}) \\
        \semsAGU{\newvar{ v:=e}{c}} &=& (\lambda \sigma' \in \Sigma. [\sigma' | v:\sigma v])_{\dag} (\semAGU{c}[\sigma|v:\semsAGU{e}]) \\
        \semsAGU{\catch{c_{0}}{c_{1}}} &=& \semAGU{c_{1}}_{+} (\semsAGU{c_{0}}) \\
        \semsAGU{\while{b}{c}} &=& \SUP F^{i} \bottom \\ \\
        F \;\omega\;\sigma &=& \begin{cases}
                                \omega_{\ast}(\semsAGU{c}) & \text{si } \semsAGU{b} \\
                                \sigma & \text{c.c}
                              \end{cases}
      \end{eqnarray*}
      \begin{eqnarray*}
        \\ \\
        \Sigma' &=& \Sigma \cup \{\abort\} \times \Sigma \\
        \semAGU{\_} &\in& \lang{comm} \rightarrow \Sigma \rightarrow \Sigma^{'}_{\bot} \\
        f_{\ast}, f_{+}, f_{\dag} &\in& \Sigma^{'}_{\bot} \rightarrow \Sigma^{'}_{\bot} \\ \\ 
        f_{\ast} x &=& \begin{cases}
                        f \sigma & \text{si } x = \sigma \in \Sigma \\
                        x & \text{c.c}
                      \end{cases} \\
        f_{+} x &=& \begin{cases}
                      f \sigma & \text{si } x = \lang{\abort, \sigma} \in \{\abort\} \times \Sigma\\
                      x & \text{c.c}
                    \end{cases} \\
        f_{\dag} x &=& \begin{cases}
                        \lang{\abort, f \sigma} & x = \lang{\abort, \sigma} \\
                        f x & x \in \Sigma \\
                        \bot & x = \bot
                      \end{cases} \\
      \end{eqnarray*}
    \end{multicols}

  \subsection*{Variables libres y asignables}
    $\begin{array}{lllllll}
      & FV(\skipp) &=& \emptyset  &\; FA(\skipp) &=& \emptyset \\
      & FV(v:=e) &=& \{v\} \cup FV(e)  &\; FA(v:=e) &=& \{v\} \\
      & FV(c_{0};c_1) &=& FV(c_0) \cup FV(c_{1})  &\; FA(c_0;c_1) &=& FA(c_0) \cup FA(c_1) \\
      & FV(\ifthenelse{b}{c_0}{c_1}) &=& FV(b) \cup FV(c_0) \cup FV(c_1)  &\; FA(\ifthenelse{b}{c_0}{c_1}) &=& FA(c_0) \cup FA(c_1)\\
      & FV(\while{b}{c}) &=& FV(b) \cup FV(c) &\; FA(\while{b}{c}) &=& FA(c) \\
      & FV(\newvar{v:=e}{c}) &=& FV(e) \cup (FV(c) - \{v\})  &\; FA(\newvar{v:=e}{c}) &=& FA(c) - \{v\} \\
    \end{array}$

  \subsection*{Teorema de Coincidencia (TC)}
    \PN Si dos estados $\sigma$ y $\sigma'$ coinciden en las variables libres de $c$, entonces da lo mismo evaluar $c$ en $\sigma$ o $\sigma'$.
    \begin{enumerate}
      \item $\forall \omega \in FV(c) . \; \sigma \omega = \sigma' \omega$ entonces
        \begin{itemize}
          \item $\semsAGU{c} = \bot = \semsp{c}$ o
          \item $\semsAGU{c} \neq \bot \neq \semsp{c}$ y $\semAGU{c} \sigma \omega = \semAGU{c} \sigma' \omega$
        \end{itemize}
      \item Si $\semsAGU{c} \neq \bot$, entonces $\forall \omega \notin FA(c) . \; \semAGU{c} \sigma \omega = \sigma \omega$
    \end{enumerate}
  
  \subsection*{Teorema de Renombre (TR)}
    \PN No importa el nombre de las variables utilizadas en las declaraciones de variables locales, es decir, las ligadas.

    \vspace{3mm}
    \PN $u \notin FV(c) - \{v\} \Rightarrow \semAGU{\newvar{u := e}{c/v\rightarrow u}} = \semAGU{\newvar{u := e}{c}}$

  \subsection*{Teorema de Sustitución (TS)}
    \PN Si $\delta$ es inyectiva sobre $FV(c)$ y $\forall \omega \in FV(c) . \; \sigma(\delta \omega) = \sigma' \omega$ entonces:
    \begin{itemize}
      \item $\semsAGU{c/\delta} = \bot = \semsp{c}$ o
      \item $\semsAGU{c/\delta} \neq \bot \neq \semsp{c}$ y $\semsAGU{c/\delta} (\delta \omega) = \semsp{c} \omega$
    \end{itemize}
    
  \subsection*{Lema de Sustitución (LS)}
    \PN Sea $FV(c) \subseteq V \subseteq \lang{var}$ tal que $\delta$ es inyectiva sobre $V$ y $\forall \omega \in V . \; \sigma(\delta \omega) = \sigma' \omega$ entonces:
    \begin{itemize}
      \item $\semsAGU{c/\delta} = \bot = \semsp{c}$ o
      \item $\semsAGU{c/\delta} \neq \bot \neq \semsp{c}$ y $\semsAGU{c/\delta} (\delta \omega) = \semAGU{c} \sigma' \omega$
    \end{itemize}
    
  \subsection*{\underline{Semántica Denotacional} (Completa)}
    \begin{eqnarray*}
      \Omega &\approx& (\Sigma' + \mathbb{Z} \times \Omega + \mathbb{Z} \to \Omega)_{\bot} \\
      \lterm &=& \psi \cp \lbot \cp \iota_0 \cp \lnorm \in \Sigma \rightarrow \Omega \\
      \labort &=& \psi \cp \lbot \cp \iota_0 \cp \labnorm \in \Sigma \rightarrow \Omega \\
      \lout &=& \psi \cp \lbot \cp \iota_1 \in \mathbb{Z} \times \Sigma \rightarrow \Omega \\
      \lin &=& \psi \cp \lbot \cp \iota_2 \in (\mathbb{Z} \rightarrow \Sigma) \rightarrow \Omega \\
      \bot_{\Omega} &=& \psi(\bot) \in \Omega \\ \\
      f_{\ast}, f_{+}, f_{\dag} &\in& \Omega \rightarrow \Omega \\
    \end{eqnarray*}
    \begin{multicols}{2}
      \begin{eqnarray*}
        \semsAGU{\skipp} &=& \lterm \sigma \\
        \semsAGU{\fail} &=& \labort \sigma \\
        \semsAGU{v := e} &=& \lterm [\sigma | v : \semsAGU{e}] \\
        \semsAGU{!e} &=& \lout(\semsAGU{e}, \lterm \sigma) \\
        \semsAGU{?v} &=& \lin(\lambda n \in \mathbb{Z}. \lterm [\sigma | v : n]) \\
        \semsAGU{\ifthenelse{b}{c_{0}}{c_{1}}} &=& \begin{cases}
          \semsAGU{c_0} \qquad \text{si } \semsAGU{b} \\
          \semsAGU{c_1} \qquad c.c \\
        \end{cases}\\
        \semsAGU{c_{0};c_{1}} &=& \semAGU{c_1}_{\ast} (\semsAGU{c_0}) \\
        \semsAGU{\catch{c_0}{c_1}} &=& \semAGU{c_1}_{+} (\semsAGU{c_0}) \\
        \semsAGU{\newvar{ v:=e}{c}} &=& (\lambda \sigma' \in \Sigma. [\sigma' | v:\sigma v])_{\dag} (\semAGU{c}[\sigma|v:\semsAGU{e}]) \\
        \semsAGU{\while{b}{c}} &=& \SUP F^{i} \bottom \\ \\
        F \;\omega\;\sigma &=& \begin{cases}
          \omega_{\ast}(\semsAGU{c}) \qquad \text{si } \semsAGU{b} \\
          \lterm \sigma \qquad\;\;\;\; \text{c.c}
          \end{cases}
      \end{eqnarray*}
      \begin{eqnarray*}
        \\
        f_{\ast} x &=& \begin{cases}
                        \bot_{\Omega} & x = \bot_{\Omega} \\
                        f \sigma & x = \lterm \sigma \\
                        \labort \sigma & x = \labort \sigma \\
                        \lout(n, f_{\ast} \omega) & x = \lout(n, \omega) \\ 
                        \lin(f_{\ast} \cp g) & x = \lin g 
                      \end{cases} \\ \\
          f_{+} x &=& \begin{cases}
                        \bot_{\Omega} & x = \bot_{\Omega} \\
                        \lterm \sigma & x = \lterm \sigma \\
                        f \sigma & x = \labort \sigma \\
                        \lout(n, f_{+} \omega) & x = \lout(n, \omega) \\ 
                        \lin(f_{+} \cp g) & x = \lin g 
                      \end{cases} \\ \\
                      f_{\dag} x &=& \begin{cases}
                        \bot_{\Omega} & x = \bot_{\Omega} \\
                        \lterm(f \sigma) & x = \lterm \sigma \\
                        \labort(f \sigma) & x = \labort \sigma \\
                        \lout(n, f_{\dag} \omega) & x = \lout(n, \omega) \\
                        \lin(f_{\dag} \cp g) & x = \lin g 
                      \end{cases}
      \end{eqnarray*}
    \end{multicols}

  \subsection*{\underline{Semántica Operacional}}
    \begin{multicols}{2}
      \[\begin{array}{cll}
        \noindent
        \overline{\lang{\skipp, \sigma} \rightarrow \sigma} \\ \\
        
        \overline{\lang{v := e, \sigma} \rightarrow [\sigma | v : \semsAGU{e}]} \\ \\ \\
        
        \lang{c_0, \sigma} \rightarrow \sigma' \\
        \overline{\lang{c_0;c_1, \sigma} \rightarrow \lang{c_1, \sigma'}} \\ \\
        \lang{c_0, \sigma} \rightarrow \lang{c'_0, \sigma'} \\
        \overline{\lang{c_0;c_1, \sigma} \rightarrow \lang{c'_0; c_1, \sigma'}} \\ \\
        
        (\semsAGU{e} = V) \\
        \overline{\lang{\ifthenelse{e}{c}{c'}, \sigma} \rightarrow \lang{c, \sigma}} \\ \\ \\
      \end{array}\]
      \[\begin{array}{cll}
        \noindent
        \\
        (\semsAGU{e} = F) \\
        \overline{\lang{\ifthenelse{e}{c}{c'}, \sigma} \rightarrow \lang{c', \sigma}} \\ \\ \\

        (\semsAGU{e} = F) \\
        \overline{\lang{\while{e}{c}, \sigma} \rightarrow \sigma} \\ \\
        (\semsAGU{e} = V) \\
        \overline{\lang{\while{e}{c}, \sigma} \rightarrow \lang{c; \while{e}{c}, \sigma}} \\ \\ \\

        \lang{c, [\sigma | v : \semsAGU{e}]} \rightarrow \sigma' \\
        \overline{\lang{\newvar{v := e}{c}, \sigma} \rightarrow [\sigma' | v : \sigma v]} \\ \\
        \lang{c, [\sigma | v : \semsAGU{e}]} \rightarrow \lang{c', \sigma'} \\
        \overline{\lang{\newvar{v := e}{c}, \sigma} \rightarrow \lang{\newvar{v := \sigma' v}{c'}, [\sigma' | v : \sigma v]}} \\ \\ \\
      \end{array}\]
    \end{multicols}

    \PN \underline{Definición:} Por \textit{ejecución} entendemos una secuencia $c_0 \rightarrow c_1 \rightarrow ...$ maximal, esto es, que no puede prolongarse más de lo que está. Dicha ejecución es infinita o termina en una configuración terminal $\sigma$. Si la ejecución es infinita, decimos que diverge y escribimos $c \uparrow$.
    \[
    \{\!\!\{c\}\!\!\}_{\sigma} = \begin{cases}
      \bot \qquad \text{si } \lang{c, \sigma} \uparrow \\
      \sigma' \qquad \text{si existe } \sigma' \text{ tal que } \lang{c, \sigma} \rightarrow^{\ast} \sigma'
    \end{cases}
    \]

    \subsubsection*{Con fallas}
      \begin{multicols}{2}
        \[\begin{array}{cll}
          \overline{\lang{\fail, \sigma} \rightarrow \lang{\abort, \sigma}} \\ \\ \\

          \lang{c_0, \sigma} \rightarrow \lang{\abort, \sigma'} \\
          \overline{\lang{c_0;c_1, \sigma} \rightarrow \lang{\abort, \sigma'}} \\ \\ \\

          \lang{c_0, \sigma} \rightarrow \sigma' \\
          \overline{\lang{\catch{c_0}{c_1}, \sigma} \rightarrow \sigma'} \\ \\ \\
        \end{array}\]
        \[\begin{array}{cll}
          \lang{c_0, \sigma} \rightarrow \lang{\abort, \sigma'} \\
          \overline{\lang{\catch{c_0}{c_1}, \sigma} \rightarrow \lang{c_1, \sigma'}} \\ \\ \\
          \lang{c_0, \sigma} \rightarrow \lang{c'_0, \sigma'} \\
          \overline{\lang{\catch{c_0}{c_1}, \sigma} \rightarrow \lang{\catch{c'_0}{c_1}, \sigma'}} \\ \\ \\

          \lang{c, \sigma} \rightarrow \lang{\abort, \sigma'} \\
          \overline{\lang{\newvar{v := e}{c}, \sigma} \rightarrow \lang{\abort, [\sigma' | v : \sigma v]}}
        \end{array}\]
      \end{multicols}

    \begin{eqnarray*}
      \{\!\!\{c\}\!\!\}_{\sigma} = \begin{cases}
        \bot \qquad\qquad\qquad  \text{si } \lang{c, \sigma} \uparrow \; (\uparrow: \text{ejecución infinita}) \\
        \sigma' \qquad\qquad\qquad \text{si existe } \sigma' \; \text{tal que } \lang{c, \sigma} \rightarrow^{\ast} \sigma' \\
        \lang{\abort, \sigma'} \qquad \text{si existe } \sigma' \; \text{tal que } \lang{c, \sigma} \rightarrow^{\ast} \lang{\abort, \sigma}'
      \end{cases}
    \end{eqnarray*}

    \PN \underline{PROPIEDADES:}
    \subsubsection*{\textbf{Lema 1:}}
      \begin{enumerate}
        \item Si $\lang{c_0, \sigma} \rightarrow^{\ast} \sigma'$ entonces $\lang{c_0;c_1, \sigma} \rightarrow^{\ast} \lang{c_1, \sigma'}$
        \item Si $\lang{c, [\sigma | v : \semsAGU{e}]} \rightarrow^{\ast} \sigma'$ entonces $\lang{\newvar{v := e}{c}, \sigma} \rightarrow^{\ast} [\sigma' | v : \sigma v]$
        \item Si $\lang{c, [\sigma | v : \semsAGU{e}]} \rightarrow^{\ast} \lang{c', \sigma'}$ entonces $\lang{\newvar{v := e}{c}, \sigma} \rightarrow^{\ast} \lang{\newvar{v := \sigma' v}{c'}, [\sigma' | v : \sigma v]}$
      \end{enumerate}
      \PN Prueba:
      \begin{enumerate}
        \item Supongamos $G_0 = \lang{c_0, \sigma} \RA \sigma'$, y que la ejecución $G_0 \RA \sigma'$ tiene $n$ pasos, es decir:
        \[
          G_0 \rightarrow G_1 \rightarrow ... G_n = \sigma'
        \]
        \PN Probaremos por inducción en $n$

        \underline{Caso base:} $n = 1$ Surge directamente de la regla:
        $\begin{array}{cll}
          \lang{c_0, \sigma} \rightarrow \sigma' \\
          \overline{\lang{c_0;c_1, \sigma} \rightarrow \lang{c_1, \sigma'}}
        \end{array}$

        \underline{Caso inductivo:} Tomamos como HI que a) vale para ejecuciones $\leq n$.
        \PN Supongamos que tenemos una ejecución de $n$ pasos, $G_0 \rightarrow G_1 \rightarrow ... G_n = \sigma'$.
        \PN Sea $G_1 \RA \sigma'$, de $n-1$ pasos, entonces por HI tenemos que $\lang{c_{0}^{1};c_1, \sigma^{1}} \RA \lang{c_1, \sigma'}$
        \PN Luego, utilizando la segunda regla de $;$ obtenemos:
        $\begin{array}{cll}
          \lang{c_0, \sigma} \rightarrow \lang{c_{0}^{1}, \sigma^{1}} \\
          \overline{\lang{c_0;c_1, \sigma} \rightarrow \lang{c_{0}^{1}; c_1, \sigma^{1}}}
        \end{array}$
        \PN Finalmente, $\lang{c_0;c_1, \sigma} \RA \lang{c_1, \sigma'}$

        \item
        \item
      \end{enumerate}

    \subsubsection*{\textbf{Lema 2:}}
      \begin{enumerate}
        \item $\lang{c, \sigma} \rightarrow \sigma' \Rightarrow \semsAGU{c} = \sigma'$
        \item $\lang{c, \sigma} \rightarrow \lang{c', \sigma'} \Rightarrow \semsAGU{c} = \semAGU{c'}\sigma'$
      \end{enumerate}
      \PN Prueba:

    \subsubsection*{\textbf{Lema 3:}} $\semsAGU{c} = \sigma' \Rightarrow \lang{c, \sigma} \rightarrow^{\ast} \sigma'$
      \vspace{3mm}
      \PN Prueba: Por inducción en la estructura de $c$.

    \subsubsection*{\textbf{Teorema:}} Para todo comando $c$ se tiene $\{\!\!\{c\}\!\!\} = \semAGU{c}$
      \vspace{3mm}
      \PN Prueba:
        \begin{itemize}
          \item Supongamos $\se{c}\s = \s'$, entonces por el \textbf{Lemma 3} $\{\!\!\{c\}\!\!\}\s = \semAGU{c}\s$.
          \item Supongamos $\se{c}\s = \bot$, entonces probaremos por el absurdo que $\la c, \s \ra \uparrow$
            \PN Supongamos que $\exists \s'$ tal que $\la c, \s \ra \to^{\ast} s'$. Aplicando sucesivamente el \textbf{Lemma 2}, llegamos a que $\se{c}\s = \s'$, lo cual es absurdo.
        \end{itemize}

    \subsubsection*{Teoremas de Coincidencia, Renombre y Sustitución para la semántica operacional}
      \PN Todos estos teoremas son válidos para la semántica operacional también. Se prueban sencillamente utilizando el teorema anterior, que dice que para todo comando c vale $\{\!\!\{c\}\!\!\} = \semAGU{c}$.