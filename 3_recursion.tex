\section{Recursión}
\begin{itemize}
  \item \textbf{Orden parcial:} Relación reflexiva, antisimétrica y transitiva.
  \item \textbf{Espacio ordenado de funciones:} \poset{Y}{\leq_Y} \entonces \poset{X \rightarrow Y}{\leq} donde $f \leq g \sii \forall x \in X. f x \leq g x$
  \item \textbf{Lifting:} \poset{X}{\leq_{X}} \entonces \poset{X_{\bot}}{\leq} donde $x \leq y \sii x \leq_{X} y \vee x = \bot$
    \PN Ejemplo: $\mathbb{Z}_{\bot}$
    \[
      \begin{array}{crrrclccc}
          \ldots & -3 & -2 & -1& 0 &1& 2& 3& \ldots \\
                      &      &  \ldots  &  \backslash   &  | & / & \ldots \\
                      &     &      &     & \perp
      \end{array}
    \]
  \item \textbf{Infinito:} \poset{X}{\leq_{X}} \entonces \poset{X^{\infty}}{\leq} donde $x \leq y \sii x \leq_{X} y \vee y = \infty$
    \PN Ejemplo: $\mathbb{N}_{\bot}$
    \[
      \begin{array}{c}
      \infty \\
        \vdots \\
        3 \\
        2 \\
        1 \\
        0
      \end{array}
    \]
  \item \textbf{Supremo:} Sea $Q \subseteq P$ donde \poset{P}{\leq}, el supremo se define:
    \begin{itemize}
      \item $\forall q \in Q. q \leq sup(Q)$
      \item $\forall p \in P.(\forall q \in Q. q \leq p) \Rightarrow sup(Q) \leq p$
    \end{itemize}
  \item \textbf{Cadenas:} $p_0 \leq p_1 \leq p_2 \dotsc$
    \begin{itemize}
      \item Interesantes: si $\{p_0, p_1, p_2, \dotsc\}$ es infinita.
      \item No interesantes: si $\{p_0, p_1, p_2, \dotsc\}$ es finita o repite infinitamente un elemento.
    \end{itemize}
  \item \textbf{Predominios:} es un poset donde todas las cadenas (interesantes) tienen supremo.
    \PN Si $Y$ es predominio entonces $X \rightarrow Y$ también lo es.
  \item \textbf{Dominios:} es un predominio con elemento mínimo.
  \PN Si $D$ es dominio entonces $X \rightarrow D$ también lo es.
  \item \textbf{Monotonía:} Sean \poset{P}{\leq_P} y \poset{Q}{\leq_Q}, y $f \in P \rightarrow Q$, f es \textit{monótona} si:
    \[
      x \leq_P y \Rightarrow f x \leq_Q f y \qquad (\text{preserva orden})
    \]
  \item \textbf{Continuidad:} Sean P, Q con $\leq_P, \leq_Q$ y $sup_P, sup_Q$ predominios y $f \in P \rightarrow Q$, se dice que $f$ es \textit{continua} si preserva supremos de cadenas, es decir, si $p_0 \leq_P p_1 \dotsc \leq_P p_n$ entonces el supremo $sup_Q(\{f p_i | i \in \mathbb{N}\})$ existe y $sup_Q(\{f p_i | i \in \mathbb{N}\}) = f sup_P(\{p_i | i \in \mathbb{N}\})$
  \item \textbf{Funciones Estrictas:} Sean $D, D'$ dominios con $\bot, \bot'$ respectivamente. Se dice que la función $f \in D \rightarrow D'$ es \textit{estricta} si $f$ preserva el elemento mínimo, es decir, $f \bot = \bot'$.
\end{itemize}

\PN \underline{PROPIEDADES:}
  \begin{itemize}
    \item \textbf{Proposición 1:} Si $f$ es monótona, f aplicada a los elementos de una cadena devuelve una cadena.
    \item \textbf{Proposición 2:} Si $f$ es monótona, entonces f preserva el supremo de cadenas no interesantes.
    \item \textbf{Proposición 3:} Si la función $f \in P \Rightarrow Q$ entre predominios es monótona entonces $sup_Q(\{f p_i | i \in \mathbb{N}\})$ existe y $sup_Q(\{f p_i | i \in \mathbb{N}\}) \leq_Q f \; sup_Q(\{p_i | i \in \mathbb{N}\})$
    \item \textbf{Proposición 4:} Si $f$ es continua, entonces $f$ es monótona.
      \PN La inversa, es decir, $f$ monótona entonces $f$ continua, solo vale para las cadenas no interesantes. Para las interesantes vale $sup_Q(\{f p_i | i \in \mathbb{N}\}) \leq_Q f sup_P(\{p_i | i \in \mathbb{N}\})$
    \item \textbf{Corolario:} Sean P, Q con $\leq_P, \leq_Q$ y $sup_P, sup_Q$ predominios y $f \in P \rightarrow Q$ monótona, entonces f es continua sii si para toda cadena interesante $p_0 \leq_P p_1 \dotsc \leq_P p_n \leq_P \dotsc$, la desigualdad $f \; sup_P(\{p_i | i \in \mathbb{N}\}) \leq sup_Q(\{f p_i | i \in \mathbb{N}\})$ también vale.
  \end{itemize}

  \subsubsection*{TEOREMA DEL MENOR PUNTO FIJO}
    \PN \textbf{Teorema:} Sea $D$ un dominio, y $F \in D \rightarrow D$ continua, entonces $sup(F^i \bot)$ existe y es el menor punto fijo de $F$.
    \PN \textbf{Prueba:} Como $\bot$ es el elemento mínimo, $\bot \leq F\ \bot$. Como $F$ es continua, $F$ es monótona. Aplicando $F$ a ambos lados obtenemos
    \[ F\ \bot \leq F\ (F\ \bot) = F^2\ \bot\]
    
    \PN Iterando esto obtenemos $\bot \leq F\ \bot \leq F^2\ \bot \leq F^3\ \bot \leq \ldots$, es decir que $\{F^i\ \bot|i \in \mathbb N\}$ es una cadena y por lo tanto el supremo $x = \sup(\{F^i\ \bot|i \in \mathbb N\})$ existe.
    
    \PN Veamos que es punto fijo de $F$, es decir, que $F\ x = x$:
    \[
      \begin{array}{rcl}
        F\ x &=& F\ \sup(\{F^i\ \bot|i \in \mathbb N\}) \\
               &=& \sup(\{F\ (F^i\ \bot)|i \in \mathbb N\}) \\
               &=& \sup(\{F^{i+1}\ \bot|i \in \mathbb N\}) \\
               &=& \sup(\{F^i\ \bot|i \in \mathbb N\}) \\
               &=& x
      \end{array}
    \]
    
    \PN Veamos que es el menor de ellos. Sea $y$ punto fijo de $F$, es decir $F\ y = y$. Veamos que $x \leq y$. Claramente $\bot \leq y$ por ser elemento mínimo. Como $F$ es monótona, se obtiene $F\ \bot \leq F\ y = y$. Iterando, obtenemos $F^i\ \bot \leq y$ para todo $i$. Es decir, $y$ es cota superior de la cadena $\{F^i\ \bot|i \in \mathbb N\}$. Como el supremo es la menor de esas cotas,

    \[
      \begin{array}{rcl}
        x &=& \sup(\{F^i\ \bot|i \in \mathbb N\}) \\
               & \leq & y
      \end{array}
    \]