\section{Cálculo Lambda}
  \[
    \begin{array}{rcll}
      \expr &::= && \text{término lambda, o expresión} \\
          &|& \var & \text{variable} \\
          &|& \expr \expr & \text{aplicación, el primero es el operador y el segundo el operando} \\
          &|& \lambda \var.\expr & \text{abstracción o expresión lambda} \\
    \end{array}
  \]

  \begin{eqnarray*}
    FV(v) &=& \{v\} \\
    FV(e_0 e_1) &=& FV(e_0)\cup FV(e_1) \\
    FV(\lambda v.e) &=& FV(e) - \{v\}
  \end{eqnarray*}
        
  \PN También se define sustitución:
  \begin{eqnarray*}
    \Delta &=& \var \to \expr \\
    \_/\_ &\in& \expr \times \Delta \to \expr \\
    v/\delta &=& \delta v \\
    (e_0 e_1)/\delta &=& (e_0/\delta) (e_1/\delta) \\
    (\lambda v.e)/\delta &=& \lambda v'.(e/[\delta|v:v']) \\
    && \text{ donde } v' \not\in \bigcup_{w \in FV(e) - \{v\}} FV(\delta w)
  \end{eqnarray*}
        
  \begin{property} \hfill
    \begin{enumerate}
    \item si para todo $w \in FV(e)$, $\delta w = \delta' w$ entonces
      $(e/\delta) = (e/\delta')$
    \item sea $i$ la sustitución identidad, entonces $e/i = e$.
    \item $FV(e/\delta) = \bigcup_{w\in FV(e)} FV(\delta w)$
    \end{enumerate}
  \end{property}

  \PN Definiciones:
  \begin{itemize}
    \item Expresión cerrada: sin variables libres
    \item Forma canónica: abstracciones $(\lambda x.e)$
    \item \textit{Redex}: expresión de la forma $(\lambda v.e) e'$.
    \item Expresion lambda cerrada: $\lambda v.e$ tal que $FV(e) = \{v\}$
    \item Formal normal: expresión sin rédices. No necesariamente son abstracciones, por ejemplo, $(\lambda x.y) (\lambda z.z)$ $\beta$-contrae a $y$ es es formal normal pero no canónica.
  \end{itemize}
        
  \paragraph{Renombre ($\alpha$)}
    La operación de cambiar una ocurrencia de la expresión lambda
      $\lambda v.e$ por $\lambda v'.(e/v \to v')$ donde
      $v' \not\in FV(e) - \{v\}$ se llama renombre o cambio de variable
      ligada.

  \paragraph{Contracción ($\beta$)}  
    Es la aplicación de una función
      $(\lambda v.e)$ a su argumento $e'$. Debe calcularse reemplazando las
      ocurrencias libres de $v$ en $e$ por $e'$, es decir $(e/v\to e')$.
        
  \begin{theorem}{Church-Rosser.}
    Si $e \totrans e_0$ y $e \totrans e_1$, entonces existe $e'$ tal que
    $e_0 \totrans e'$ y $e_1 \totrans e'$.
  \end{theorem}
        
  \begin{corollary}
    Salvo renombre, toda expresión tiene a lo sumo una forma normal.
  \end{corollary}

  \vspace{3mm}
  \PN NO TODA EXPRESIÓN TIENE FORMA NORMAL (contraejemplos).
  \begin{itemize}
    \item Sea $\Delta = (\lambda x.xx)$, $\Delta \Delta$ no es forma normal
    \item Sea $\Delta' = (\lambda x.xxy)$, $\Delta' \Delta'$ no es forma normal
  \end{itemize}
  
  \begin{property}
    Una aplicación cerrada no puede ser forma normal.
  \end{property}
  \begin{proof}
    \PN Sea $e$ una aplicación cerrada, es decir $e = e_0 e_1 ...$. Si $e_0$ fuera una variable, $e$ no sería cerrada, por ende $e_0$ es una abstracción. Por lo tanto $e$ contiene el redex $e_0 e_1$ y por esto no es normal.
  \end{proof}

  \begin{corollary}
    Si una expresión cerrada, es forma normal entonces es forma canónica.
  \end{corollary}
  \PN Al revés no vale, contrajemplo $\lambda x. (\lambda y. y) x$.

  \subsubsection*{Diferencias entre $\rightarrow$ y $\Rightarrow$}
    \begin{enumerate}
      \item solo se evaluan expresiones cerradas
      \item es determinística
      \item no busca formas normales sino formas canónicas
    \end{enumerate}

  \subsection{Evaluación}
    \PN En este tipo de semántica operacional se describen la relación entre los términos y sus valores, que también son términos, formas canónicas. Llamamos $\Rightarrow$ a esta relación. Cuando decimos que $e \Rightarrow z$ se cumple, estamos diciendo que existe un árbol de derivación que prueba $e \Rightarrow z$.
    
    \vspace{3mm}
    \PN Puede ocurrir que la evaluación eager no termine mientras que la normal si, por ejemplo: $(\lambda x. \lambda y. y) (\Delta\Delta) \Rightarrow_{N} \lambda y.y$, mientras que en eager diverge.
    
    \paragraph{Regla $\eta$}
      Un $\eta$-redex es una expresión de la forma $\lambda v.e\, v$ donde $v \not\in  FV(e)$. Gracias a la
        regla $\beta$, uno obtiene que $(\lambda v.e\,v) e'$ contrae a $e\, e'$ para toda expresión $e'$. Si
        uno asume que toda expresión lambda denota una función, $\lambda v.e\,v$ y $e$ parecen
        comportarse extensionalmente igual: cuando se las aplica a $e'$, ambas dan $e e'$.
        Esto motiva la regla $\eta$:

    \begin{prooftree}
      \AxiomC{\ }
      \RightLabel{si $v \not\in FV(e)$}
      \UnaryInfC{$(\lambda v.e v) \to e$}
    \end{prooftree}
    
  \subsection{Semántica denotacional}
    \PN Sea $\dinf \cong [\dinf \to \dinf]$, donde $\dinf \rightarrow \dinf$ se refiere al espacio de funciones continuas donde todas tiene punto fijo, y sean:
      \[
        \begin{array}{rl}
        \phi \,\in\, & \dinf \to  [\dinf \to  \dinf]\\
        \psi \,\in\, & [\dinf \to  \dinf] \to  \dinf
        \end{array}
      \]
      los isomorfismos tales que
      \begin{align*}
      \phi . \psi &= id \\
      \psi . \phi &= id
      \end{align*}

      \begin{eqnarray*}
        \text{Ambiente: } \Env &=& \var \to \dinf \\
        \eta &\in& \Env \\
        \llbracket\_\rrbracket &\in& \expr \to  Env \to  \dinf \\ \\
        \llbracket v\rrbracket\eta &=& \eta\, v \\
        \llbracket e_0\, e_1\rrbracket\eta &=& \phi(\llbracket e_0\rrbracket \eta) (\llbracket e_1\rrbracket\eta) \\
        \llbracket\lambda x.e\rrbracket\eta &=& \psi(\lambda d \in \dinf. \llbracket e\rrbracket[\eta |v:d]) \\
      \end{eqnarray*}        
        
    \begin{theorem}{Coincidencia} Si $\eta\, w$ = $\eta'\, w$ para todo $w \in FV(e)$,
      entonces $\llbracket e\rrbracket \eta = \llbracket e\rrbracket \eta'$.
    \end{theorem}
        
    \begin{theorem}{Sustitución} Si $\llbracket \delta\, w\rrbracket\eta = \eta' w$ para todo
      $w \in FV(e)$, entonces $\llbracket e/\delta\rrbracket \eta =\llbracket e\rrbracket\eta'$.
    \end{theorem}
    
    \begin{theorem}{Sustitución Finita} $\llbracket e/v_1\to e_1, \ldots,
      v_n\to e_n\rrbracket\eta = \llbracket e\rrbracket[\eta|v_1:
      \llbracket e_1\rrbracket\eta|\ldots|v_n:\llbracket e_n\rrbracket\eta]$.
    \end{theorem}
    
    \begin{theorem}{Renombre} Si $v' \not\in FV(e)-\{v\}$, entonces
      $\llbracket\lambda v'.(e/v\to v')\rrbracket = \llbracket\lambda
      v.e\rrbracket$.
    \end{theorem}
        
    \begin{property}(correctitud de la regla $\beta$): $\llbracket(\lambda v.e)\, e'\rrbracket = \llbracket e/v\to e'\rrbracket$.
    \end{property}
        
    \begin{property}
      (correctitud de la regla $\eta$): Si $v \not\in FV(e)$, entonces
      $\llbracket\lambda v.e v\rrbracket = \llbracket e\rrbracket$.
    \end{property}

    \begin{corollary}
      Si $e \to^{\ast} e'$, entonces $\se{e} = \se{e'}$
    \end{corollary}
        
    \PN Asumiremos que $\llbracket\Delta \Delta\rrbracket\eta = \bot$.

    \subsubsection{Normal}
      \begin{eqnarray*}
        D &=& V_\bot, \text{ donde } V \cong [D \to  D] \\ \\
        \phi &\in& V \to  [D \to  D] \\
        \psi &\in& [D \to  D] \to  V \\ \\ 
        \lambda d &\in& D. \bot \text{ bottom de } D \rightarrow D \\
        \psi(\lambda d &\in& D. \bot) \text{ bottom de V pero no de D} \\ \\
        \phi_{\botbot} &\in& D \to  [D \to  D] \\
        \iota_\bot . \psi &\in& [D \to  D] \to  D \\ \\
        \Env &=& \var \to  D \\
        \llbracket\_\rrbracket &\in& \expr \to  Env \to  D \\ \\
        \llbracket v \rrbracket\eta &=& \eta v \\
        \llbracket e_0\, e_1\rrbracket\eta &=& \phi_{\botbot} (\llbracket e_0\rrbracket\eta) (\llbracket e_1\rrbracket\eta) \\
        \llbracket\lambda x.e\rrbracket\eta &=& (\iota_\bot . \psi)(\lambda d \in D. \llbracket e\rrbracket[\eta|v:d])
      \end{eqnarray*}

      \PN Los teoremas antes vistos junto con la regla $\beta$ siguen valiendo pero no la regla $\eta$, pues $\se{\lambda y. \Delta\Delta y} = \psi(\lambda d \in D.\bot)$ mientras que $\se{\Delta\Delta} = \bot$.
    
    \subsubsection{Eager}
      \begin{eqnarray*}
        D &=& V_\bot, \text{ donde } V \cong [V \to D] \\ \\
        \phi &\in& V \to [V \to D] \\
        \psi &\in& [V \to D] \to V \\ \\
        \phi_{\botbot} &\in& D \to [V \to D] \\
        \iota_\bot . \psi &\in& [V \to D] \to D \\ \\
        Env &=& \var \to V \\
        \llbracket\_\rrbracket &\in& \expr \to Env \to D \\ \\
        \llbracket v\rrbracket\eta &=& \iota_\bot (\eta v) \\
        \llbracket e_0\, e_1\rrbracket\eta &=& (\phi_{\botbot} (\llbracket e_0\rrbracket\eta))_{\botbot}\, (\llbracket e_1\rrbracket\eta)\\
        \llbracket\lambda x.e\rrbracket\eta &=& (\iota_\bot . \psi)(\lambda z \in V. \llbracket e\rrbracket[\eta|x:z])
      \end{eqnarray*}

      \PN Los teoremas antes vistos (salvo sustitución) siguen valiendo pero no las reglas $\beta$, ya que $\se{(\lambda y \lambda x.x)(\Delta \Delta)}$ diverge en eager y no da $\se{\lambda x. x}$ como esperaría la regla y $\eta$, ya que $\se{\lambda y. \Delta\Delta y} = \psi(\lambda d \in D.\bot)$ mientras que $\se{\Delta\Delta} = \bot$.